\documentclass[10pt,twocolumn,letterpaper]{article}
\usepackage[UTF8]{ctex}
\usepackage{cite}

\begin{document}
    \title{联邦学习中的通信优化}
    \author{叶茂青,王珺}

    \maketitle
    \begin{abstract}
    随着机器学习的不断发展,机器学习所需要的数据量也不断增加,但客户对于数据隐私的需要,使得机器学习所需的数据获取难度很高,联邦学习试图在保护用户数据的同时,建立出能满足用户需求的模型。但大规模的联邦学习训练对带宽要求极高,对于个人设备,高昂的通信成本更是限制了复杂模型的使用,因此,提高联邦学习中客户端与服务器端的通信效率极为重要。本文总结了在联邦学习中常用的通信优化方法。
    \end{abstract}
    
    \section{联邦学习的介绍}
    xxx

    \section{梯度压缩}
    针对
    \begin{table}[]
    \begin{tabular}{|l|l|l|}
    \hline
                            & Works        \\ \hline
    Gradient quantization   & \begin{tabular}[c]{@{}l@{}}Wen et al.\cite{Wen}\\ Seide et al.\cite{Seide2014}\\ Zhou et al.\cite{Zhou}\end{tabular}                                         \\ \hline
    Gradient sparsification & \begin{tabular}[c]{@{}l@{}}Storm\cite{International}\\ Dryden et al.\cite{Dryden2016}\\ Aji \& Heafielf\cite{Aji2017}\\ Chen et al.\cite{Chen}\end{tabular} \\ \hline
    \end{tabular}
    \end{table}

    \subsection{Gradient quantization}
    Gradient quantization的思路主要是xxx

    \subsection{Gradient sparsification}
    Gradient sparsification的思路主要是xxx


    {\small
        \bibliographystyle{IEEEtran} 
        \bibliography{assignment1}
    }
\end{document}