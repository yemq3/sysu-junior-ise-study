\documentclass[10pt,onecolumn,letterpaper]{article}

\usepackage{xeCJK}
\setCJKmainfont[AutoFakeSlant=0.25]{Noto Sans Mono CJK SC}
\setCJKsansfont[AutoFakeSlant=0.25]{Noto Sans Mono CJK SC}
\setCJKmonofont[AutoFakeSlant=0.25]{Noto Sans Mono CJK SC}

\usepackage{cite}
\usepackage{amsmath}
\usepackage{amssymb}

\begin{document}
    \title{联邦学习中的通信优化}
    \author{17363092 叶茂青,17363079 王珺}

    \maketitle
    \begin{abstract}
        联邦学习的提出使得用户可以在保证隐私的同时贡献自己的数据,但高昂的通信成本限制了联邦学习的应用,本文总结了联邦学习针对网络梯度传输所做的各种优化。
    \end{abstract}
    
    
    \section{TODO}
    problem that the research addresses

    background information and relevant references

    elements that validate the level of innovation of the research

    conceptual model, methodology or procedure that the research takes into consideration

    analysis and interpretation of the results achieved

    strengths and weaknesses of the research, the insights demonstrated

    implications for further research


    \section{联邦学习的介绍}
    联邦学习的数学模型描述在2016年由McMahan et al.\cite{McMahan2016}提出,McMahan et al.同时也提出了联邦学习区别于分布式学习的四大难点:
    \begin{itemize}
        \item Non-IID
        \item Unbalanced
        \item Massively distributed
        \item Limited communication
    \end{itemize}
    由于联邦学习的应用场景中,客户端多为个人设备,通信成本高且通信质量也不能保证


    \section{梯度压缩}
    针对
    \begin{table}[htb]
    \begin{tabular}{|l|l|l|}
    \hline
                            & Works        \\ \hline
    Gradient quantization   & \begin{tabular}[c]{@{}l@{}}Wen et al.\cite{Wen}\\ Seide et al.\cite{Seide2014}\\ Zhou et al.\cite{Zhou}\end{tabular}                                         \\ \hline
    Gradient sparsification & \begin{tabular}[c]{@{}l@{}}Storm\cite{International}\\ Dryden et al.\cite{Dryden2016}\\ Aji \& Heafielf\cite{Aji2017}\\ Chen et al.\cite{Chen}\end{tabular} \\ \hline
    \end{tabular}
    \end{table}

    \subsection{Gradient quantization}
    Gradient quantization的思路主要是将原来32位浮点数存储的梯度信息量化为更低准确度的存储表示。这方面的研究有Seide et al.\cite{Seide2014}提出的1-bit SGD、Alistarh et al.\cite{Alistarh}提出的QSGD、Wen et al.\cite{Wen}提出的TernGrad、Zhou et al.\cite{Zhou}提出的DoReFa-Net、Storm et al.\cite{International}提出的xxx
    
    Gradient quantization的难点在于如何在压缩梯度后依然能保证网络的收敛,对此Seide et al.\cite{Seide2014}使用Error Feedback的方法,将每次的量化误差保存下来并再次反馈到量化函数中

    虽然1-bit SGD在实验观察下依旧可以保持收敛,但在其他条件下的收敛性依旧存疑,Alistarh et al.\cite{Alistarh}提出的QSGD则用数学证明了其在任意函数下的收敛性,在实验的表现中,使用QSGD对梯度进行4Bit或8Bit量化,无论是在resnet还是lstm上的精度都和32BitSGD没有太大区别,同时实验也提出了卷积层对于量化的敏感性或许要更高。QSGD允许用户在通信量和训练时间进行权衡

    Wen et al.\cite{Wen}提出的TernGrad将梯度为2Bit,使用-1,0,1三个值表示梯度。其数学表示为
    \[
        \tilde{\boldsymbol{g}}_{t}=\operatorname{ternarize}\left(\boldsymbol{g}_{t}\right)=s_{t} \cdot \operatorname{sign}\left(\boldsymbol{g}_{t}\right) \circ \boldsymbol{b}_{t}
    \]
    where
    \[
        s_{t} \triangleq \max \left(a b s\left(\boldsymbol{g}_{t}\right)\right) \triangleq\left\|\boldsymbol{g}_{t}\right\|_{\infty}
    \]
    其中$b_t$满足伯努利分布
    \[
        \left\{\begin{array}{l}
        P\left(b_{t k}=1 \mid \boldsymbol{g}_{t}\right)=\left|g_{t k}\right| / s_{t} \\
        P\left(b_{t k}=0 \mid \boldsymbol{g}_{t}\right)=1-\left|g_{t k}\right| / s_{t}
        \end{array}\right.
    \]
    为了避免某些梯度过大导致绝大部分梯度被量化为0,作者引入layer-wise ternarizing,逐层的对梯度进行量化,同时对于不同层之间梯度方差的差异,作者引入gradient clipping,用方差信息对每一层的梯度进行裁剪。

    作者使用在MNIST和CIFAR-10上训练的卷积网络进行验证,相比32BitSGD,使用TernGrad并没有减慢收敛的速度,且精度损失也在1\%以内。

    \subsection{Gradient sparsification}
    Gradient sparsification的思路主要是xxx
    \cite{}
    \cite{Lin2017}


    {\small
        \bibliographystyle{IEEEtran} 
        \bibliography{assignment1.bib}
    }
\end{document}